% Note for next  update
%
\documentstyle[11pt]{article}
%\documentstyle[11pt,epsf]{article}
%\documentstyle[12pt,epsf]{article}

\begin{document}
%\baselineskip 24pt

\begin{center}
{\Large\bf Treecode User's Guide}\\
\bigskip
{\large Junichiro Makino}\\
\bigskip

Department of Information Science and Graphics,\\
College of Arts and Sciences,\\
University of Tokyo,\\
Komaba 3--8--1, Meguro-ku, Tokyo 153, Japan\\
Phone : 81--3--3465--3925\\
Fax : 81--3--3465--3925 (same as phone)\\
E-mail : makino@kyohou.c.u-tokyo.ac.jp\\
\end{center}

\begin{center}
Version 1 : Dec 21, 1998\\
Version 1.1 : Dec 22, 1998\\
Version 1.1A : Jan 1st, 1999\\
Version 1.1B : Feb 9th, 1999\\
\end{center}

\def\nbody{{\tt nbody}}
\def\vcode{{\tt v\_code}}
\bigskip
\begin{abstract}

This guide gives a brief desciption of the program \nbody, a C++
version of treecode and its GRAPE-4 variant.
\end{abstract}


\section{Changes}

\subsection{Version 1 to 1.1}

The {\tt qsort(3)} routine is replaced by a more usual quicksort
(taken from Handbook of Algorithms and Data Structures,
http://www.thunderstone.com/jump/demos/algorithms/hbook.html). The
performance problem disappered. 

\subsection{Version 1.1 to 1.1A}

Note that this corresponds to the change of the version 1.0 to 1.1 of
the program itself.

The main routine is rewrited to use commandline arguments instead of
the parameter file.

\subsection{Version 1.1A to 1.1B}

No change in functionality. Just made some bug fix so that the code
can be compiled with g++ 2.8.1. Previous versions works only with
2.7.2 or older. 

\section{What is \nbody ?}

\nbody is a C++ implementation of Barnes-Hut treecode. This program is
intended as the replacement for the old FORTRAN version of treecode
developed by the author (\vcode). The algorithm used in \nbody and
\vcode\ are essntially the same. The main goal in the desing of \nbody 
is to minimize the memory usage.

\section{Installing \nbody}

\subsection{Requirements}

\nbody requires a reasonably new C++ compiler (currently, I have
tested it only with g++ on Digital UNIX V4.0), which support 64 bit
integer (as {\tt long}). You may be able to change the source code to
support 32-bit platform, but I have never tried.

Standard Makefile for \nbody assumes that PGPLOT is available. It's
easy to edit Makefile so that you can run \nbody without PGPLOT.

\subsection{Getting \nbody}

In Komaba astrophysics computer network, you can find the most recent
version of \nbody as
\begin{verbatim}
~makino/src/C++tree/export.tar.gz
\end{verbatim}
You can just copy it to your local directory.

It's also available at my Web cite
\begin{verbatim}
http://grape.c.u-tokyo.ac.jp/~makino/softwares/C++tree/
\end{verbatim}


\subsection{Compiling \nbody}

First go to the directory to which yo install \nbody. Then extract the 
files from the tar file you've got.

Then, you may need to edit the Makefile according to your need. The
default Makefile is designed for Digital UNIX platform with GRAPE-4
installed, and it looks like

\begin{verbatim}
PGPLOT_DIR = /usr/local/pgplot5.2
#PGPLOT_DIR = /data6/makino/PDS/pgplot_linux/pgplot
PGINCLUDE =  -I$(PGPLOT_DIR)
CFLAGS = $(FLOAT_OPTIONS) -DREAL=double -O4 -ffast-math -funroll-loops

# Add -DNOGRAPHICS for installation without PGPLOT
 CPPFLAGS =   -DNOGRAPHICS -DTREE $(PGINCLUDE) -pg $(FLOAT_OPTIONS) -O4   -ffast-math -funroll-loops
PGLIB     =  -L$(PGPLOT_DIR) -g3 -O4 -lcpgplot -lpgplot -lX11 -lUfor -lfor -lFutil -lm -lprof1 -lpdf
G4LIB =  -L/usr6/kawai/pub/lib -lg4a -lrt -lm
H3LIB = -L/usr2/makino/src/harp3board -lharp3  -lrt -lm 

LIBOBJS = pgetopt.o gravity.o BHtree.o second.o
LIBOBJSG4 = pgetopt.o  gravity_g4.o BHtree_g4.o second.o

EXPORT_FILES = BHtree.h nbody_particle.h vector.h \
               BHtree.C	gravity.C nbody.C pgetopt.C\
	       second.c \
	       Makefile nbody_guide.tex\
               samplein sampleparm samplelog  samplelog_g4 

export: export.tar.gz
	ls -al export.tar.gz
export.tar.gz: $(EXPORT_FILES)
	tar cvf export.tar  $(EXPORT_FILES)
	gzip export.tar 
gravity.o : gravity.C  nbody_particle.h
	g++  $(CPPFLAGS)  -c gravity.C
gravity_g4.o : gravity.C  nbody_particle.h
	g++  -DHARP3 $(CPPFLAGS)  -c gravity.C
	mv gravity.o gravity_g4.o

nbody.o : nbody.C  nbody_particle.h
	g++  -c    $(CPPFLAGS)  nbody.C 
nbody : nbody.C  nbody_particle.h $(LIBOBJS)
	g++  -DEVOLVE    $(CPPFLAGS) -o nbody nbody.C $(LIBOBJS) 
nbody_g4 : nbody.C  nbody_particle.h $(LIBOBJSG4)
	g++  -DEVOLVE    $(CPPFLAGS) -o nbody_g4 nbody.C $(LIBOBJSG4) $(PGLIB) $(G4LIB)
nbody_h3 : nbody.C  nbody_particle.h $(LIBOBJSG4)
	g++  -DEVOLVE    $(CPPFLAGS) -o nbody_h3 nbody.C $(LIBOBJSG4) $(H3LIB)

BHtree.o : BHtree.C  nbody_particle.h BHtree.h 
	g++    $(CPPFLAGS) -c BHtree.C 
BHtree_g4.o : BHtree.C  nbody_particle.h BHtree.h 
	g++ -DHARP3   $(CPPFLAGS) -c BHtree.C
	mv BHtree.o  BHtree_g4.o 
pgetopt.o : pgetopt.C  
	g++  $(CPPFLAGS)  -c pgetopt.C
samplelog: sampleparm nbody
	nbody < sampleparm > samplelog
samplelog_g4: sampleparm nbody_g4
	nbody_g4 < sampleparm > samplelog_g4





\end{verbatim}

If you do not have PGPLOT, add {\tt -DNOGRAPHICS} to CPPFLAGS. If you
have one, specify the installed directory at {\tt PGPLOT\_DIR}.

If you have GRAPE-4, check {\tt G4LIB} so that it points to the right
directory. In addition, target name for HARP-3 (GRAPE-4 with
Turbochannel driver) version ({\tt nbody\_h3}) is also provided. This
one uses {\tt H3LIB} instead of {\tt G4LIB}. To use this one in
Komaba, change {\tt PGPLOT\_DIR} to {\tt /usr/local/pgplot} or specify 
{\tt -NOGRAPHICS} in {\tt CPPFLAGS}

To compile \nbody, just type
\begin{verbatim}
make nbody
\end{verbatim}
or, for GRAPE-4 version, 
\begin{verbatim}
make nbody_g4
\end{verbatim}
If you've met by an error message, well, good luck! Please let me know 
how you fixed the problem.

\section{Running \nbody}
Parameters are taken from the command line.
The complete list of options (which you can find in the {\tt main()}
routine) is:

\begin{verbatim}
list of options
-i        name of snapshot input file       (no default)
-o        name of snapshot output file      (default: no output)
-d        timestep                          (default: 0.015625)
-D        time interval for snapshot output (default: 1)
-T        time to stop integration          (default: 10)
-e        softening parameter epsilon2      (default: 0.025)
-t        opening angle theta               (default: 0.75)
-n        ncrit for Barnes' vectrization    (default: 1024)
          (ONLY used with GRAPE/HARP inplementation)
-w        window size for PGPLOT snapshot plot (default: 10)
-c        flag for collision run
-x        relative position vector for collision run (no default)
-v        relative velocity vector for collision run (no default)
-s        scale factor for position scaling (default: 1)
-S        scale factor for velocity scaling (default: 1)
-h        print this help
\end{verbatim}

In the following, I give somewhat more detailed explanation:
\begin{description}
\item{\bf -i:} Specify the name of input snapshot file. This is 
required. The file format is nemo\footnote{Nemo is available at
Peter Teuben's Web page, {\tt
http://bima.astro.umd.edu/nemo/}}
stoa format. 
\item{\bf -o:} Specify the name of output snapshot file. This  is 
optional. The file format is nemo stoa. When not specified, no output
snapshot is made.
\item{\bf -d:} The step size for time integration. The default value
is 1/64.

\item{\bf -D:}The time interval to output snapshot. The default value
is 1. 

\item{\bf -T:}The time to stop integration. The default value is 10.

\item{\bf -e:}The softening parameter.  The default value
is 1/40. 

\item{\bf -t:}The opening angle for the force calculation.  The default value
is 0.75. 

\item{\bf -n:}The critical number of particles in the cell, used in
Josh Barnes' vectrization scheme.  The default value is 1024.  Only
used with GRAPE/HARP implementation. {\tt Nbody} uses the standard
particle-based scheme on general-purpose computer. 

\item{\bf -w:} The size of PGPLOT window. The default value is
10. When compiles without {\tt -NOGRAPHICS}, {\tt nbody} creates a
PGPLOT window and displays the projection of the system to the x-y
plane. This option specifies the size of the region displayed.

\item{\bf -c:} If specified, collision of two identical system is set
up. Default is no collision.

\item{\bf -x:} When -c option is given, specify the initial separation 
as, for example,  {\tt -x 10 0 0} (x, y, and z coordinates). 
\item{\bf -v:} Similar to x option, but for initial relative
velocity. 

\item{\bf -s:}Multiply the initial position  of each particle by this parameter. Default 
is 1. 
\item{\bf -S:}Multiply the initial velocity of each particle  by this parameter. Default 
is 1. 

\item{\bf -h:} Print a short summary of the command line options.

\end{description}

The file {\tt samplein} is a Plummer model with 4096 particles, in
Heggie units.  You can try a test run with
\begin{verbatim}
nbody -i samplein
\end{verbatim}
You may compare the log output (in {\tt stdout}) with
file {\tt samplelog}. The sample log file is obtained on an Alpha
clone with 600 MHz EV56 chip (164LX motherboard).

Some additional timing information goes to standard error.

\section{Memory requirement}

According to the measurement, \nbody requires about $8+200*(N/10^6)$ MB of
main memory, where $N$ is the number of particles. GRAPE-4 version
would require additional memory for the library, which I believe a few 
tens of MBs. Thus, you should be able to run a $10^6$-body simulation
on a machine with 256MB memory. 

\section{Features and Bugs}

\subsection{Integer word length}

The current implemntation of {\tt nbody} relies on the Morton key
encoded into a single {\tt long int} word. Actual number of bits used
(per dimension) is defined in the file BHtree.h as integer constant
{\tt default\_key\_length}. Set this value not exceeding 10 on
machines {\tt long} is of 32-bit length. On machines with 64-bit {\tt
long}, you can set its value to up to 21.

The value of {\tt default\_key\_length} determines the maximimum
possible depth of the tree. Unless you are using really very large
number of particles (say, $10^7$ or more), the maximum of 10 levels is 
probably enough.

If the maximum depth is not suffucient to resolve the distribution of
particles, {\tt nbody} just stops subdividing the cell there. As a
result, the lowest cell may contain rather large number of
particles. This does not cause any integration error, but causes some
performance degradation.

\subsection{Memory management for tree}

Currently, {\tt nbody} stops if the memory allocated for tree turned
out to be insufficient to build the tree. One could change the program 
so that either it reallocates a  larger amount memory or stop to
refining the tree at that point. It's not clear which approach is
better, and neither option is implemented yet.

\section{Description of the algorithm implemented and the data
structure}

Not completed yet...



\section{Things to be done}

\subsection{Version 1}
\begin{itemize}
\item Currently, \nbody is rather slow. The force calculation routine runs
at the speed essentially the same as that of \vcode, but tree
constraction phase take roughly twice longer time. This is because the 
{\tt qsort(3)} routine used in BHtree.C is less than optimal. A
handcoded version would significantly improve the performance.
\end{itemize}

\subsection{Version 1.1}
\begin{itemize}
\item The source code needs some cleaning up... It's mostly not
documented and not easily understandable.

\item snapshot input and output should be taken from standard input
and output. However, then, what shall we do with the log output? Of
course, the starlab  format solves the problem, but it's a bit too
large... 
\end{itemize}
\end{document}









